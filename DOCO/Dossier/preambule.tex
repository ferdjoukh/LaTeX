\documentclass[10pt]{article} % use larger type; default would be 10pt

\usepackage[utf8]{inputenc} % set input encoding (not needed with XeLaTeX)
\usepackage[french]{babel}
\usepackage[french]{translator}
%%% Examples of/home/ferdjoukh/Documents/Expes02-12/montrer_refs_exploser/allsolu.tex
% These packages are optional, depending whether you want the features they provide.
% See the LaTeX Companion or other references for full information.

%%% PAGE DIMENSIONS
\usepackage{geometry} % to change the page dimensions
%\geometry{a4paper} % or letterpaper (US) or a5paper or....
% \geometry{margin=2in} % for example, change the margins to 2 inches all round
% \geometry{landscape} % set up the page for landscape
%   read geometry.pdf for detailed page layout information

\usepackage{graphicx} % support the \includegraphics command and options
\usepackage[parfill]{parskip} % Activate to begin paragraphs with an empty line rather than an indent
\usepackage[cmex10]{amsmath}
\usepackage{array}

%%% PACKAGES
\usepackage{booktabs} % for much better looking tables
\usepackage{array} % for better arrays (eg matrices) in maths
\usepackage{paralist} % very flexible & customisable lists (eg. enumerate/itemize, etc.)
\usepackage{verbatim} % adds environment for commenting out blocks of text & for better verbati
\usepackage{subfig} % make it possible to include more than one captioned figure/table in a single float
% These packages are all incorporated in the memoir class to one degree or another...

%\geometry{hmargin=3cm,vmargin=3cm}
\usepackage{fancyhdr} % This should be set AFTER setting up the page geometry
%\usepackage[2cm,headings]{fullpage}

\pagestyle{fancy}
\fancyhf{}
\renewcommand{\headrulewidth}{0pt} 
\renewcommand{\footrulewidth}{0.1pt} 
\lfoot{Dossier professionnel}
\cfoot{\thepage}
\rfoot{Adel Ferdjoukh}


\usepackage{multicol,multirow}

\pagenumbering{arabic}
\pagestyle{fancyplain}
\thispagestyle{empty}



\usepackage{ucs}
\usepackage{inputenc}
\usepackage{verbatim}
\usepackage{moreverb}
\usepackage{url}
\usepackage{amsmath,amssymb}
\usepackage[colorlinks=none,linkcolor=black,pdfborder={0 0 0}]{hyperref}

\usepackage{xcolor}

\usepackage{tikz}
\usepackage{gnuplot-lua-tikz}
\usetikzlibrary{calc}


%%%%%%%%%%%%%%%%%%%%%%%%%%%%%%%%%%%%%%%%%%%%%%%%%%%%%%%%%
%                                                       %
%    Grimm                                              %    
%                                                       %
%%%%%%%%%%%%%%%%%%%%%%%%%%%%%%%%%%%%%%%%%%%%%%%%%%%%%%%%%
\newcommand{\grimm}{$\mathcal{G}$\textsc{rimm}}
\newcommand{\grimmtext}{$\mathcal{G}$ene\textsc{r}ating \textsc{i}nstances of \textsc{m}eta-\textsc{m}odels}
\DeclareRobustCommand{\grrimm}{$\mathcal{G}$\reflectbox{\textsc{r}}\textsc{rimm}}
\DeclareRobustCommand{\grrimmtext}{$\mathcal{G}$enerating \reflectbox{\textsc{r}}andomized and \textsc{r}elevant \textsc{i}nstances of \textsc{m}eta-\textsc{m}odels}


%%%%%%%%%%%%%%%%%%%%%%%%%%%%%%%%%%%%%%%%%%%%%%%%%%%%%%%%
%%%%%%%%%%%                                   %%%%%%%%%%
%%%%%%%%%                                       %%%%%%%%
%%%%%%%           Pour les courbes               %%%%%%%
%%%%%%%%%                                       %%%%%%%%
%%%%%%%%%%%                                   %%%%%%%%%%
%%%%%%%%%%%%%%%%%%%%%%%%%%%%%%%%%%%%%%%%%%%%%%%%%%%%%%%% 


\usepackage{tikz}
\usepackage{gnuplot-lua-tikz}

\newcommand{\MonChapitre}[2]{%
    \chapter[#1]{#2}
    \minitoc
    \addcontentsline{lof}{chapter}{%
    Chapitre \thechapter : #2 \vspace{10pt}}
}


\newcommand{\MonAnnexe}[2]{%
    \chapter[#1]{#2}
    \minitoc
    \addcontentsline{lof}{chapter}{%
    Annexe \thechapter : #2 \vspace{10pt}}
}

\usepackage{ucs}
\usepackage{url}

\usepackage{amsthm}
\usepackage{enumitem}
\usepackage{listings}
\theoremstyle{definition}
\newtheorem{example}{Exemple}
\newtheorem{definition}{Définition}
\newtheorem{remark}{Remarque}

\usepackage{tabularx}
\usepackage{pdfpages}

\newcommand{\topfigrule}{\hrule\kern-0.4pt\relax}

\newcommand{\nb}[2]{
    \fbox{\bfseries\sffamily\scriptsize#1}
    {\sf\small\textit{\textcolor{red}{#2}}}
}

\newcommand\cn[1]{\nb{CN}{#1}}
\newcommand\ac[1]{\nb{AC}{#1}}
\newcommand\ad[1]{\nb{Adel}{#1}}
\newcommand\adel[1]{\nb{Adel}{#1}}
\newcommand\Adel[1]{\nb{Adel}{#1}}


\usepackage{natbib}
\usepackage{bibentry}
\bibliographystyle{plainnat}