\section{Activités d'enseignement}

Cette section détaille les activités d'enseignement que j'ai assuré au sein de l'Université de Montpellier au cours de ma mission complémentaire d’enseignement et de l'Université de Nantes en ma qualité d'attaché temporaire d’enseignement et de recherche. Le total des heures d'enseignement assurées jusqu'ici est de {\bf 650 heures}.

%%%%%%%%%%%%%%%%%%%%%%%%%%%%%%%%%%%%%%%%%%%%%%%%%%%%%%%%%
%                                                       %
%   2014-2015                                           %    
%                                                       %
%%%%%%%%%%%%%%%%%%%%%%%%%%%%%%%%%%%%%%%%%%%%%%%%%%%%%%%%%
\subsection{2014-2015}

Durant l'année universitaire 2014-2015, je suis intervenu dans 3 matières différentes. Cet enseignement concernait les étudiants de L2 (faculté des sciences et polytech Montpellier) et de Master 2 en génie logiciel (master Aigle).   

\begin{tabular}{m{5cm}||m{1.3cm}m{1.7cm}m{2cm}m{3cm}} 
    \hline
    {\bf Matière}                      & {\bf Niveau} & {\bf Étudiants} & {\bf Nature} & {\bf Volume horaire} \\
    \hline
    Bases de données                   & L2 & $1\times30$ & TD/TP & $24 + 9$ \\
    Programmation Orientée Objet       & L2 & $1\times30$ & TD/TP & $12 + 16.5$ \\
    Ingénierie dirigée par les modèles & M2 & $1\times40$ & TD/TP & $7.5 +15$ \\
    \cline{4-5}
     & & & {\bf Total} & $82$h \\
    \hline
\end{tabular}

{\bf Ingénierie dirigée par les modèles}

    \begin{itemize}
        \item J'ai mis en place un tutoriel/cours sur l’environnement EMF (Eclipse Modelling Framework).
        \item Je me suis occupé de réaliser des fiches de TP pour le semestre entier.
        \item J'ai réalisé un sujet de projet de fin de semestre et je me suis occupé de l'évaluation des étudiants.
    \end{itemize}

{\bf Bases de données} J'ai participé aux corrections des contrôles continus et de l'examen final.
 
{\bf Programmation objet} J'ai réalisé des sujets de contrôle continu en TD et en TP

%%%%%%%%%%%%%%%%%%%%%%%%%%%%%%%%%%%%%%%%%%%%%%%%%%%%%%%%%
%                                                       %
%   2015-2016                                           %    
%                                                       %
%%%%%%%%%%%%%%%%%%%%%%%%%%%%%%%%%%%%%%%%%%%%%%%%%%%%%%%%%
\subsection{2015-2016}

L'année universitaire 2015-2016 a été l'occasion découvrir de nouvelles matières et des classes différentes et diversifiées. J'ai assuré un enseignement dans 3 matières différentes pour des classes de L1, L3 et master 2.

\begin{tabular}{m{5cm}||m{1.3cm}m{1.5cm}m{2.5cm}m{3cm}} 
\hline
Programmation impérative           & L1 & $1\times30$ & TD/TP & $3+18$ \\
Réseaux                            & L3 & $1\times30$ & TD/TP & $16.5 + 19.5$ \\
Ingénierie dirigée par les modèles & M2 & $1\times40$ & Cours/TD/TP & $3+6+12$ \\
\cline{4-5}
 & & & {\bf Total} & $78$h \\
\hline
\end{tabular} 

{\bf Ingénierie dirigée par les modèles} Mon rôle était similaire à celui de l'année précédente. J'ai par ailleurs participé à l'évaluation des étudiants en fin d'année qui s'est faite via des lectures d'articles de recherche.

{\bf Réseaux} J'ai pris par aux corrections des contrôles et des projets de TP.

{\bf Programmation impérative} J'ai réalisé un sujet de contrôle continu de TD et un autre pour le TP.

%%%%%%%%%%%%%%%%%%%%%%%%%%%%%%%%%%%%%%%%%%%%%%%%%%%%%%%%%
%                                                       %
%   2016-2017                                           %    
%                                                       %
%%%%%%%%%%%%%%%%%%%%%%%%%%%%%%%%%%%%%%%%%%%%%%%%%%%%%%%%%
\subsection{2016-2017}

Durant cette année d'ATER à l'Université de Nantes, j'ai voulu me consacrer aux classes débutantes en informatique. L'enseignement que j'assure concerne les L1 et L2.

\begin{tabular}{m{6cm}||m{1.3cm}m{1.5cm}m{2cm}m{3cm}} 
\hline
Algorithmique                            & L1 & $2\times30$ & TD/TP & $16 + 20$ \\
Algorithmique pour Biologistes           & L1 & $2\times30$ & TD/TP & $24 + 18$ \\
C2i                                      & L1 & $2\times30$ & TD/TP & $38$ \\
Bases de données                         & L1 & $1\times30$ & TD/TP & $20 + 20$ \\ 
Algorithmique et Structures de données   & L2 & $2\times30$ & TD/TP & $24+30$ \\
Création de pages web                    & L2 & $2\times15$ & TP & $15$ \\
\cline{4-5}
                                         & & & {\bf Total} & $240$h \\
\hline
\end{tabular} 

{\bf Algorithmique} J'ai réalisé le sujet de contrôle continu de TD.

{\bf Algorithmique pour biologistes} J'ai réalisé 2 sujets de contrôle continu de TD et proposé plusieurs sujets de projets.

{\bf C2i} J'ai participé à la correction des examens et des comptes rendus de TP des étudiants. 

{\bf Bases de données} J'ai participé à la corrections des examens et des projets des étudiants.

{\bf Algorithmique et Structures de données} J'ai participé à la corrections des examens et des projets des étudiants.

{\bf Création de pages web} J'ai participé à la corrections des projets et des TP notés.

%%%%%%%%%%%%%%%%%%%%%%%%%%%%%%%%%%%%%%%%%%%%%%%%%%%%%%%%%
%                                                       %
%   2017-2018                                           %    
%                                                       %
%%%%%%%%%%%%%%%%%%%%%%%%%%%%%%%%%%%%%%%%%%%%%%%%%%%%%%%%%
\subsection{2017-2018}

\begin{tabular}{m{6cm}||m{1.3cm}m{1.5cm}m{2cm}m{3cm}} 
\hline
Algorithmique                          & L1 & $2\times35$ & TD/TP & $32 + 20$ \\
Algorithmique pour Biologistes         & L1 & $3\times35$ & TD/TP & $27 + 18$ \\
Méthodologie du Travail Universitaire  & L1 & $1\times35$ & TD/TP & $38$ \\
Bases de données                       & L1 & $1\times30$ & TD/TP & $16 + 12$ \\ 
Introduction au développement logiciel & L1 & $2\times30$ & TD/TP & $26 + 26$ \\ 
Programmation Orientée Objet           & L2 & $1\times30$ & TP & $12$ \\
Programmation Avancée en C             & M1 & $2\times30$ & TP & $12$ \\
\cline{4-5}
                                       & & & {\bf Total} & $226$h \\
\hline
\end{tabular} 

{\bf Algorithmique} J'ai élaboré $2$ sujets de contrôle continu de TD et proposé plusieurs thèmes de projets.

{\bf Algorithmique pour biologistes} J'ai proposé plusieurs thèmes de projets et participé aux évaluations des étudiants (examen et projets).

{\bf Méthodologies du travail universitaire} J'ai donné un cours sur le {\it scepticisme scientifique} et organisé l'évaluation des étudiants (présentation orale en groupe). 

{\bf Bases de données} J'ai proposé un sujet pour le projet ({\it Game of Thrones}). De plus j'ai participé à la corrections des CC et examen.

{\bf Introduction au développement logiciel} Ma participation dans cette matière inédite à l'université de Nantes fut importante:

\begin{itemize}
\item Co-organisation du module.
\item Participation à la réalisation du matériel du cours (partie test unitaires).
\item Réalisation de la partie distancielle (questionnaires, git, html).
\item Réalisation de 2 fiches de TD (test unitaires et découpage des projets informatiques en fichiers).
\item Réalisation de 3 fiches de TP (test unitaires et découpage des projets informatiques en fichiers $\times 2$).
\item Élaboration d'un sujet de contrôle continu de TD.
\end{itemize}

{\bf Programmation Orientée Objet} J'ai participé à la correction des projets des étudiants.

{\bf Programmation Avancée en C} J'ai participé à la correction des projets des étudiants.

\subsection{Récapitulatif}

En l'espace de quatre années, j'ai assuré un total d'environ $650$ heures d'enseignement dans $2$ universités différentes: Montpellier, puis Nantes.

J'ai pris soin de diversifier les matières pour couvrir un spectre très important de l'informatique. De plus, mes cours ont concerné tous les niveaux d'études à l'université (de la licence $1$ au master $2$). Le bagage des étudiants est également très différent. Ainsi, j'ai assuré des cours devant les profils suivants:

\begin{itemize}
\item Licence en informatique (Licence 1, Licence 2 et Licence 3).
\item Licence généraliste en sciences (Licence 1 et Licence 2).
\item Licence en Biologie et Chimie (Licence 1).
\item Élève d'école préparatoire ($2^{eme}$ année).
\item Master en informatique (Master 2).
\item Master en bio-informatique (Master 1).
\end{itemize} 

%%%%%%%%%%%%%%%%%%%%%%%%%%%%%%%%%%%%%%%%%%%%%%%%%%%%%%%%%
%                                                       %
%   Réalisation pédagogiques                            %    
%                                                       %
%%%%%%%%%%%%%%%%%%%%%%%%%%%%%%%%%%%%%%%%%%%%%%%%%%%%%%%%%
\subsection[Faits notables]{Faits notables et réalisations de matériel pédagogique}

\subsubsection*{Matériel Pédagogiques}

\begin{itemize}
\item Ingénierie Dirigée par les Modèles 
    \begin{itemize}
    \item Élaboration de 5 fiches de TP et d'un projet pour les étudiants de Master 2 génie logiciel de l'Université de Montpellier.
    \item Élaboration d'un cours-tutoriel sur l'environnement Eclipse/EMF pour des étudiants du même Master.
    \end{itemize}

\item Introduction au Développement Logiciel
    \begin{itemize}
        \item Participation à la réalisation du cours (partie tests unitaires, outil jasmine).
        \item Confection de 3 fiches de TD pour les étudiants en informatique en première année de Licence.
        \item Réalisation de 3 fiches de TP.
    \end{itemize}
\end{itemize}

\subsubsection*{Sujets d'examens et de projets}

J'ai réalisé une dizaine de sujets de contrôle continu ou bien de projet de fin de semestre.

\begin{itemize}
\item {\it Ingénierie Dirigée par les Modèles}, 1 sujet (Projet, 2014).

\item {\it Programmation Orientée Objet}, 2 sujet (TD, TP, 2015).

\item {\it Programmation impérative}, 1 sujet (TD, 2016).

\item {\it Algorithmique}, 3 sujets (TD, 2016,2017).

\item {\it Algorithmique pour biologiste}, 2 sujets (TD, 2016).

\item {\it Introduction au développement logiciel}, 1 sujet (TD, 2018).

\item {\it Bases de données}, 1 sujet (TP, 2018).
\end{itemize}

\subsubsection*{Innovation pédagogique}

Durant la préparation des cours ou la réalisation des sujets de projets ou d'examen, je prends soin de réfléchir à des thèmes et activités qui intéresseront les étudiants. Parmi les initiatives prises, je cite les plus importantes: 

\begin{itemize}
\item Proposer des sujets qui correspondent le mieux au profil des étudiants. Ainsi, pour mes étudiants de licence en biologie et chimie, j'ai proposé un examen lié à la saponification et un TP sur les tableaux périodiques en javascript.
\item Proposer des sujets d'actualité ou ludique. J'ai par exemple élaboré un sujet lié à la série {\it Game of Thrones} en bases de données.
\item Utiliser des supports médiatiques et journalistiques. J'ai par exemple proposé la critique d'un documentaire ({\it Le mythe des géants}) pour illustrer le concept de pseudo-science.
\item TD sous forme de workshop. J'ai proposé aux étudiants de travailler en groupe pour découper un projet informatique et réaliser chacun sa part. Ceci les a amenés à se questionner sur l'utilité de Git.
\item Concours de programmation. Nous avons organisé (avec 3 collègues) le concours du meilleur projet informatique en L1. La cérémonie de remise de prix s'est déroulée en présence du doyen de la faculté des sciences. 
\end{itemize}

\subsubsection*{Responsabilités diverses}  

J'ai effectué une dizaine d'heures référentielles. Il s'agissait d'aider les étudiants de première année de licence à s'orienter, au cours d'entretiens individuels.  