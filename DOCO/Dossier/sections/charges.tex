\section{Charges collectives}

Durant ma thèse de doctorat à l'université de Montpellier, j'ai également pris part à l'animation scientifique de l'école doctorale et du laboratoire lirmm. Voici le résumé de ces activités.

\begin{itemize}
\item Co-organisation de la $22^{i\grave{e}me}$ journée des doctorants de l'école doctorale I2S de Montpellier (Information, Structures, Systèmes), {\bf Doctiss}. 

\bigskip

\begin{tabular}{m{3cm}|m{4cm}|m{3cm}|m{3cm}}
\hline
Événement & Nature & Lieu & Quand ? \\
\hline 
Doctiss & Journée scientifique & Montpellier & 02 à 06/14 \\
\hline
\end{tabular}

\bigskip
\noindent
{\it Charge de travail:} Nous nous réunissions (groupe de $10$ doctorants) 1h30 par semaine durant $4$ mois pour préparer l’événement (appels à présentations et à posters, site web, inscriptions des participants, organisation du déjeuner, cadeau aux participants). L’événement en lui même a duré une journée entière (4 juin 2014). Il consistait en une dizaine de présentations et une vingtaine de posters de domaines différents. 

\medskip
\noindent
{\it Tâches assignées:} Je me suis occupé des inscriptions des participants, de la co-gestion du site web et de la relecture des propositions de présentations et de posters. 

\bigskip

\item Co-organisation des séminaires des doctorants du laboratoire lirmm de septembre 2014 à juillet 2015, {\bf Semindoc}.

\bigskip

\begin{tabular}{m{3cm}|m{4cm}|m{3cm}|m{3cm}} 
\hline
Semindoc & Séminaires/ demi-journées & Montpellier & 09/14 à 07/15 \\
\hline
\end{tabular}

\bigskip
\noindent
{\it Charge de travail:} Nous nous réunissions (équipe de $3$ doctorants) régulièrement pour organiser un séminaire toutes les trois semaines et deux demi-journées par an. Nous devions également un budget alloué par le laboratoire.

\medskip
\noindent
{\it Tâches assignées:} Chercher des intervenants pour proposer un séminaire, gérer le budget et diffuser des annonces de séminaires.

\item Volontariat aux conférences Ecmfa, Ecoop, Ecsa 2013 en juillet 2013. 

\end{itemize}  