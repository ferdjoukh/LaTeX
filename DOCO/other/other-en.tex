%%%%%%%%%%%%%%%%%%%%%%%%%%%%%%%%%%%%%%%%%%%%%%%%%%%%%%%%%%%%
%%%%%%%%%%%%%%%%%%%%%%%%%%%%%%%%%%%%%%%%%%%%%%%%%%%%
% Divers de recherche                              %
%%%%%%%%%%%%%%%%%%%%%%%%%%%%%%%%%%%%%%%%%%%%%%%%%%%%
%%%%%%%%%%%%%%%%%%%%%%%%%%%%%%%%%%%%%%%%%%%%%%%%%%%%%%%%%%%%
\lefttitle{notable research facts}

\begin{tabular}{r @{~$\rangle$~} p{0.75\textwidth}}

Best paper award & ICSEA 2017\\

International collaboration & involved in Disolo, a european reseach project between Nantes \& Oslo \\

Reviewing & JFPC 2014, MODELSWARD 2016 \& ISSRE 2017 \\

\end{tabular}
