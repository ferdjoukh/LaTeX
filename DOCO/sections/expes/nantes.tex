%%%%%%%%%%%%%%%%%%%%%%%%%%%%%%%%%%%%%%
%                                    %
%  Vie associative                   %
%                                    %
%%%%%%%%%%%%%%%%%%%%%%%%%%%%%%%%%%%%%%
\righttitle{Université de Nantes - laboratoire LS2N}

\smalltitle{Domaine de compétences}{black} : R\&D en MDE

\smalltitle{Intitulé de l’intervention}{black} : Création d’un fault localizer pour les modèles EMF

\bigskip

\smalltitle{Objectifs(s) :}{blue}

\begin{itemize}
\item Debugger un générateur de modèles
\item Utilisation de la programmation linéaire
\end{itemize} 

\smalltitle{Réalisation(s) :}{blue}

\begin{itemize}
\item Tiwizi : un outil pour debugger les générateurs de modèles \footnote{\url{https://github.com/ferdjoukh/tiwizi}}
\item 2 publications internationales.
\end{itemize} 

\smalltitle{Environnement(s) technique(s) :}{blue}

\begin{itemize}
\item MDE, Java
\item Programmation linéaire
\item Bash
\item Git
\end{itemize} 